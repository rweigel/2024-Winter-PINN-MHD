Active regions are the totality of observable phenomena in a 3D volume represented by the extension of the magnetic field from the photosphere to the corona, revealed by emissions over a wide range of wavelengths from radio to X-rays and γ-rays (only during flares) accompanying and following the emergence of strong twisted magnetic flux (kG, ≥ 1020 Mx) through the photosphere into the chromosphere and corona.”, as described by [van Driel-Gesztelyi and Green, 2015]. Active regions are areas of strong magnetic fields and emerge in the photosphere and as mentioned in the definition, they extend to the corona. Active regions in the photosphere can have either simple bipole patterns or more complex ones that consist of multiple bipoles that emerge in the same area. On the photosphere they appear as dark areas called sunspots and consist of a dark central area called umbra and a lighter dark coloured area called penumbra. They appear as dark patches due to the strong magnetic field, which prevents the heat transportation from the solar interior to the surface and as a result they are cooler than the ambient environment. Figure 1.1 shows an example of a complex active region as observed by HMI/ SDO. The HMI continuum is given on panel (a) and the LOS magnetogram on (b). Image source: https://helioviewer.org/, [Mu ̈ller et al., 2017].
